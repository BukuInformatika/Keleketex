\section{Pembagian bab}
Secara default pembagian bab pada latex menggunakan perintah \textit{section}, \textit{subsection}, \textit{subsubsection} dan \textit{subsubsubsection}. Untuk mengatur kedalaman suatu dokumen pada bab bab tertentu, kita dapat menggunakan perintah berikut ini pada bagian Preamble :
setcounter.secnumdepth
setcounter.tocdepth



Opsi yang digunakan pada syntax secnumdepth pada perintah verbcounter= seperti perintah diatas, berarti Anda telah merubah kedalaman bab yang Anda perbaharui sampai dengan level 5 yaitu section -- subsection -- subsubsection -- paragraph -- subparagraph. Sedangkan pada perintah dari opsi tocdepth berfungsi untuk membuat table of contents atau menampilkan kedalaman bab sampai dengan level 5, namun jika tidak di setel maka pada bagian level 3 kebawah tidak akan dapat ditampilkan pada bagian toc \ref{labelgambar}.


\begin{figure}[ht]
\centerline{\includegraphics[width=1\textwidth]{figures/capture.JPG}}
\caption{Pembagian Bab.}
\label{labelgambar}
\end{figure}


\section{Format Cetak}
Pada format LateX teks mempunyai bentuk plaintext, yang artinya teks tersebut belum diformat. Pada proses formatting teks dapat dilakukan dengan bahasa tersendiri yaitu bahasa markup. Hal paling mendasar antara lain cetak tebal, miring dan gari bawah. Cetak tebal menggunakan perintah \textit{textbf},cetak miring menggunakan perintah \textit{textit} dan garis bawah menggunakan perintah \textit{underline}.

\section{Tanda petik}
Tanda petik di Latex menggunakan petik miring dan petik satu. Petik miring biasanya berada pada sebelah angka satu di keyboard dan diakhiri petik satu. Ingat fungsi tanda petik hanya untuk melakukan quote atau pengutipan langsung. Untuk istilah bahasa inggris gunakan miring atau italic.

\begin{lstlisting}[caption=Contoh kalimat dalam tanda petik pada Latex,label={lst:tandapetik}]
`kalimat dalam tanda petik'
\end{lstlisting}

\section{Penomoran}
Perintah penomoran pada latex biasanya menggunakan format \textit{Numbering} atau format \textit{Bullets}. Perintah yang digunakan pada format Numbering adalah \textit{enumerate} sedangkan untuk Bullets yang menyerupai poin menggunakan \textit{itemize}.



\textit{Numbering} merupakan perintah yang digunakan untuk membuat daftar berurut dengan penomoran menggunakan angka (numbered list), yang biasanya diberikan pada awal baris baru. Sedangkan \textit{Bullets} atau poin adalah perintah yang digunakan untuk membuat daftar berurut dengan penomoran berupa symbol atau poin (bulleted list). Pada listing \ref{lst:penomoran} adalah perintah untuk memasukan listing pada penomoran .


\lstinputlisting[caption=Memberikan Perintah Numbering,label={lst:PenomoranNumbering}]{src/1/lstlistingNumbering.tex}
Sedangkan \textit{Bullets} atau poin adalah perintah yang digunakan untuk membuat daftar berurut dengan penomoran berupa symbol atau poin (bulleted list) \ref{lst:penomoran}.
\lstinputlisting[caption=Menambahkan kode perintah bullets,label={lst:penomoran}]{src/1/poin.tex}

\section{Kode Program}
Agar kita dapat memasukan kode program, kita dapat menggunakan perintah \textit{lstlisting}. Perintah ini  berfungsi untuk memasukkan atau menambahkan kode program apapun ke dalam file yang terpisah. Untuk memasukan perintah \textit{lstlisting} kita perlu menulis parameter \textit{caption} dan \textit{label} untuk memberikan penjelasan keterangan kode program dan sebagai sumber referensi dari label kode program.

\lstinputlisting[caption=Menambahkan kode program,label={lst:kodeprogram}]{src/1/lstlisting.tex}


\section{Costum Command}
Sesuai  dengan  namanya Costum Command, dimana ke unggulan latex ada fitur yang satu ini, Pembuat dokumen ini dapat  membuat macro untuk kebutuhan yang sifatnya spesifik dan berulang-ulang, dimana costum cummad dapat melakukan tanda bintang berjejer sebagai penanda garis.

\section{Menambahkan Spesial Karakter}
Untuk menambahkan karakter spesial pada LaTex kita dapat menggunakan tanda \textit{backslash} didepan karakter yang ingin kita tandai. Terdapat beberapa karakter yang tidak bisa langsung digunakan seperti tanda \textit{ampersand}. Selain itu format pemberian kutipan pada LaTex berbeda dengan pemberian kutipan pada editor lainnya, cara memasukkan karakter spesial menggunakan listing \ref{lst:kodespesial}
\lstinputlisting[caption=Contoh kode untuk menambahkan karakter spesial,label={lst:kodespesial}]{src/1/spesial.tex}

\section{Menambahkan Chapter}
Berikut ini merupakan langkah-langkah untuk menambahkan \textit{chapter} baru.
\begin{enumerate}
\item Pertama kita buat \textit{chapter} baru pada repositori kita di folder \verb|chapters|, seperti pada gambar \ref{fig:tambahchapter}.

\begin{figure}[!htbp]
  \centering
  \includegraphics[width=.75\textwidth]{figures/tambahchapter.png}
  \caption{Menambahkan Chapter Baru}\label{fig:tambahchapter}
\end{figure}

\item Kemudian kita tambahkan kode seperti pada \textit{listing} \ref{lst:input} yang berfungsi untuk memanggil \textit{chapter} yang baru kita tambahkan pada file \verb|main.tex| seperti pada gambar \ref{fig:inputchapter}.

\lstinputlisting[caption=Penggunaan perintah input untuk menambahkan chapter,label={lst:input}]{src/1/input.tex}

\begin{figure}[!htbp]
  \centering
  \includegraphics[width=.75\textwidth]{figures/inputchapter.png}
  \caption{Menambahkan Perintah Input Chapter}\label{fig:inputchapter}
\end{figure}

\item Terakhir, compile file \verb|main.tex| untuk melihat chapter baru yang telah kita tambahkan pada file main.pdf.
\end{enumerate}

