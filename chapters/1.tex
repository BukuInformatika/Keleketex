\section{Pembagian bab}
Pembagian bab di latex menggunakan perintah \textit{section}, \textit{subsection}, \textit{subsubsection} dan \textit{subsubsubsection}.


\section{Format Cetak}
Hal paling mendasar antara lain cetak tebal, miring dan gari bawah. Cetak tebal menggunakan perintah \textit{textbf},cetak miring menggunakan perintah \textit{textit} dan garis bawah menggunakan perintah \textit{underline}.

\section{Tanda petik}
Tanda petik di Latex menggunakan petik miring dan petik satu. Petik miring biasanya berada pada sebelah angka satu di keyboard dan diakhiri petik satu.

\begin{lstlisting}[caption=Contoh kalimat dalam tanda petik di Latex,label={lst:tandapetik}]
`kalimat dalam tanda petik'
\end{lstlisting}

\section{Penomoran}
Penomoran di latex menggunakan perintah \textit{enumerate} sedangkan untuk poin menggunakan \textit{itemize}.

\section{Karakter Khusus}


\section{Kode Program}
Kode program menggunakan \textit{lstlisting}. Jangan lupa parameter \textit{caption} dan \textit{label} senantiasa ditulis.
\lstinputlisting{src/listings.tex}
