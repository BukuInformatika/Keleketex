\section{Membuat Rumus dengan LaTex}
Sebagai aplikasi editor pengolah dokumen, LATEX memiliki kemampuan yang mampu menghasilkan dokumen berisi notasi-notasi matematis. Agar dapat menghasilkan dokumen yang berisikan notasi-notasi matematis maka kita harus berada dalam \textit{Mathematics Environtment}. Terdapat beberapa perintah yang bisa digunakan dalam membuat rumus pada latex. Kita dapat menggunakan perintah \textit{equation}, \textit{displaymath} ataupun menggunakan \$. Kita juga dapat menyelipkan rumus didalam suatu kalimat di sebuah paragraf dengan menggunakan perintah \$\$.

\section{Penulisan Notasi Matematika}
Pada latex kita dapat menuliskan suatu notasi matematika yang cukup panjang dalam suatu paragraf baru. Penulisan Notasi Matematika dalam suatu paragraf dapat dilihat pada listing \ref{lst:notasi_paragraf}.
\lstinputlisting[caption=Notasi Matematika Dalam Paragraf,label={lst:notasi_paragraf}]{src/3/notasi1.tex}

\section{Jenis Font Dalam Notasi Matematika}
Ada beberapa perintah pada yang dapat digunakan untuk mengubah jenis font notasi matematis dalam latex. Beberapa perintah tersebut dapat kita lihat pada listing \ref{lst:fontmath}.
\lstinputlisting[caption=Jenis Font Matematis,label={lst:fontmath}]{src/3/font.tex}

Hasil output : 

$\mathrm{x y z}$

$\mathsf{x y z}$

$\mathtt{x y z}$

$\mathit{x y z}$

$\mathbf{x y z}$

\section{Rumus Dasar}
Rumus dasar ini terdiri dari 3 notasi yaitu penjumlahan, pengurangan, dan perkalian. Contoh kode untuk rumus dasar bisa dilihat pada listing \ref{lst:rumus_dasar}.
\lstinputlisting[caption=Penggunaan Rumus Dasar,label={lst:rumus_dasar}]{src/1/rumus_dasar.tex}
Hasil output:

$$ a+b$$

$$ a-b$$

$$ a \times b$$

\subsection{Rumus Pecahan}
Rumus pecahan yang dimaksud adalah notasi per pada pembagian. Contoh kode untuk rumus pecahan bisa dilihat pada listing \ref{lst:rumus_pecahan}.
\lstinputlisting[caption=Penggunaan Rumus Pecahan,label={lst:rumus_pecahan}]{src/1/rumus_pecahan.tex}
Hasil output:

$$ a/b$$

$$ \frac {a}{b}$$

\subsection{Rumus Akar}
Rumus akar dapat dilihat pada listing \ref{lst:rumus_akar}.
\ref{lst:rumus_akar}.
\lstinputlisting[caption=Penggunaan Rumus Akar,label={lst:rumus_akar}]{src/1/rumus_akar.tex}
Hasil output:

$$ \sqrt[a]{b}$$

$$ \sqrt{\sqrt{a}}$$

\section{Perumusan Menggunakan Superscripts dan Subscripts}
Penulisan \textit{Supserscripts} dan \textit{Subscripts} biasanya digunakan untuk membuat sebuah rumus dengan menghasilkan pangkat diatas dan pangkat dibawah pada suatu rumus. Cara penulisan penggunaan ini adalah dengan menggunakan perintah \textbf{sp} dan perintah \textbf{sb}. Untuk contoh penerapan perintah \textit{Supserscripts} dan \textit{Subscripts} dapat kita lihat pada listing \ref{lst:sp1}.

\lstinputlisting[caption=Penggunaan Supersripts dan Subscripts,label={lst:sp1}]{src/3/sp1.tex}

Hasil output :

\begin{displaymath}
y = x\sb{1}\sp{2} + x\sb{2}\sp{2}
\end{displaymath}

Atau kita juga dapat menggunakan perintah lain seperti pada listing \ref{lst:sp2}.

\lstinputlisting[caption=Perintah Pada Superscripts dan Subscripts,label={lst:sp2}]{src/3/sp2.tex}

Hasil output :

\begin{displaymath}
f(x) = e^{x_1}
\end{displaymath}

\section {Perumusan Array dan Matriks}
Dalam LaTex, kita dapat menuliskan rumus sebuah array pada environment \textbf{tabular}. Perintah untuk membuat array dan matriks dapat kita lihat pada listing \ref{lst:array}.
\lstinputlisting[caption=Penulisan Array atau Matriks,label={lst:array}]{src/3/array.tex}

Hasil output :

\begin{displaymath}
\left (
\begin{array}{rrr}
0 & 55 & 23 \\
34 & -83 & 68 \end{array}
\right )
\end{displaymath}

Ada beberapa hal yang perlu kita ketahui dalam penulisan rumus array atau matriks ini :
\begin{itemize}
\item Penulisan array memiliki kesamaan seperti saat membuat format tabel
\item Perintah \textbf{"rrr"} berfungsi untuk menentukan posisi dari masing-masing komponen matriks tersebut
\item Tanda kurung kurawal "( )" berfungsi untuk mendefinisikan bagian kurung buka dan kurung tutup pada sebuah matriks  
\end{itemize}

\section{Perumusan Vektor}
Dalam LaTex, perumusan dengan format \textit{vektor} kita dapat menuliskannya dengan perintah seperti pada listing \ref{lst:vektor}.
\lstinputlisting[caption=Penulisan Vektor,label={lst:vektor}]{src/3/vektor.tex} 

Contoh kita akan mengubah variabel \textit{x} kedalam satuan vektor. Maka hasil outputnya adalah :
\begin{displaymath}
\vec{x} = a + b
\end{displaymath} 

\section{Kombinasi Penggunaan Rumus}
Pada section ini kita akan mempelajari bagaimana mengkombinasikan sebuah rumus dari penulisan dasar rumus Subscript, Superscript, Akar Pangkat, Pecahan dan sejenisnya. Contoh pertama dapat kita lihat pada listing \ref{lst:sigma}.
\lstinputlisting[caption=Contoh Kombinasi Rumus Sigma,label={lst:sigma}]{src/3/sigma.tex}

Hasil output :

$$\sum^{\infty}_{n=1} \frac{1}{n}$$

Setelah melihat salah satu penggunaan kombinasi rumus diatas kita bisa melakukan kombinasi rumus lainnya. Seperti yang akan diperlihatkan pada listing \ref{lst:kombinasi}

\lstinputlisting[caption=Contoh Kombinasi Rumus,label={lst:kombinasi}]{src/3/kombinasi.tex}

Hasil output :
\begin{enumerate}
\item $\sqrt{ \frac{a^2}{3b^3+1}}$
\item $\lim_{n \to \infty} \frac{1}{n}=0$
\item $\int^b_a x^2 \, dx$
\item $\lim \limits_{n \to \infty} \frac{1}{n}=0$
\item $\int \limits^b_a x^2 \, dx$
\item $\sum \limits^{\infty}_{n=1} \frac{1}{n}$
\end{enumerate}

\section{Penulisan Tata Letak Rumus}
Pada section ini kita kan mempelajari bagaimana menempatkan penulisan rumus sesuai dengan tata letak yang kita inginkan. Misalkan kita ingin membuat suatu rumus dan hasil dari rumus tersebut ingin kita letakkan di tengah, kiri atau pun kanan. Contoh penerapan tata letak dalam sebuah rumus dapat kita lihat pada listing \ref{lst:tataletak}.
\lstinputlisting[caption=Contoh Penulisan Tata Letak Dalam Rumus,label={lst:tataletak}]{src/3/tataletak.tex}

Hasil output :
\begin{enumerate}
\item $\begin{array}{ccc} (x+y)(x-y) & = & x^2-xy + yx-y^2 \\ & = & x^2-y^2 \\ (x+y)^2 & = & x^2 + 2xy + y^2 \end{array}$

\item $\begin{array}{lcr} (x+y)(x-y) & = & x^2-xy + yx-y^2 \\ & = & x^2-y^2 \\ (x+y)^2 & = & x^2 + 2xy + y^2 \end{array}$

\item $\begin{array}{rcl} (x+y)(x-y) & = & x^2-xy + yx-y^2 \\ & = & x^2-y^2 \\ (x+y)^2 & = & x^2 + 2xy + y^2 \end{array}$
\end{enumerate}

\section{Penulisan Simbol}
Banyak penulisan simbol yang dapat kita implementasikan pada latex. Beberapa contoh perintah penggunaan simbol pada latex dapat kita lihat seperti pada listing \ref{lst:simbol}
\lstinputlisting[caption=Contoh Penulisan Simbol,label={lst:simbol}]{src/3/simbol.tex}

Hasil output :
\begin{enumerate}
\item $\pi$ = Merupakan simbol \textbf{pi}	
\item $\phi$	= Merupakan simbol \textbf{phi}
\item $\rho$ = Merupakan simbol \textbf{rho}
\item $\sigma$ = Merupakan simbol \textbf{sigma}	
\item $\epsilon$ = Merupakan simbol \textbf{epsilon}
\item $\delta$ = Merupakan simbol \textbf{delta}	
\item $\theta$ = Merupakan simbol \textbf{theta}	
\item $\kappa$ = Merupakan simbol \textbf{kappa}	
\item $\alpha$ = Merupakan simbol \textbf{alpha}
\item $\beta$ = Merupakan simbol \textbf{beta}
\item $\gamma$ = Merupakan simbol \textbf{gamma}	
\item $\omega$ = Merupakan simbol \textbf{omega}	
\item $\zeta$ = Merupakan simbol \textbf{zeta}	
\item $\eta$	= Merupakan simbol \textbf{eta}
\item $\iota$ = Merupakan simbol \textbf{iota}	 
\item $\lambda$ = Merupakan simbol \textbf{lambda}	
\item $\mu$ = Merupakan simbol \textbf{mu}
\item $\nu$ = Merupakan simbol \textbf{nu}	
\item $\xi$ = Merupakan simbol \textbf{xi}
\item $\tau$ = Merupakan simbol \textbf{tau}
\item $\upsilon$ = Merupakan simbol \textbf{upsilon}	
\item $\chi$ = Merupakan simbol \textbf{chi}	
\item $\psi$ = Merupakan simbol \textbf{psi}
\end{enumerate}





